\documentclass[6pt]{article}

\usepackage[letterpaper]{geometry}
\usepackage{mathtools}
\usepackage{amsmath}
\usepackage{amssymb}
\usepackage{lmodern}
\geometry{top=1.0cm, bottom=1.0cm, left=1.0cm, right=1.0cm}
\pagenumbering{gobble}

% horizontal line spanning the whole page

\newcommand{\HL}{\par\noindent\rule{\textwidth}{0.4pt}}

\begin{document}

\begin{scriptsize}

\begin{center}
\textbf{MATH-338 Final Cheat Sheet [FRAMEWORKS]}
\end{center}

\HL

\begin{tabular}{l | l}

\parbox{0.45\textwidth}{

\begin{flushleft}
\textbf{Neyman-Pearson Hypothesis Testing}
\end{flushleft}

\begin{itemize}
\item H\textsubscript{o}: $\mu = 0$
\item H\textsubscript{a}: $\mu = n$
\item Requires a rejection region, a small area where the null hypothesis should be rejected
\item If the observed value falls in the region, H\textsubscript{a} is true, reject H\textsubscript{o}, vice versa.
\end{itemize}

\begin{flushleft}
\textbf{Fisher's Significance Testing}
\end{flushleft}

\begin{itemize}
\item Select an appropriate test
\item Set up H\textsubscript{o}
\item Calculate the theoretical proabability of the results under H\textsubscript{o} (\emph{p})
\item If \emph{p} $= \alpha \therefore$ statistically significant
\item If \emph{p} $> \alpha \therefore$ statistically insignificant
\end{itemize}

\begin{flushleft}
\textbf{POWER ANALYSIS}
\end{flushleft}

\begin{enumerate}
\item To compute the critical region:
\begin{itemize}
\item need $\alpha$, H\textsubscript{0} (value of P under H\textsubscript{0})
\item Sampling distribution of test statistic under H\textsubscript{0}
\end{itemize}
\item To computer power
\begin{itemize}
\item Need critical region, H\textsubscript{1} (value of P under H\textsubscript{1})
\item Sampling distribution of test statistic under H\textsubscript{1}
\end{itemize}
\end{enumerate}
}

&

\parbox{0.5\textwidth}{
\begin{flushright}

\begin{flushleft}
\textbf{Null Hypothesis Significance Testing}
\end{flushleft}

\begin{itemize}


\item H\textsubscript{o}: (if candy causes cancer, assume candy does not cause cancer and find counter arguments)
\item H\textsubscript{a}: $\theta [<,>,\ne] \theta_{1}$
\item Find its distribution under H\textsubscript{o}
\item Define a critical region such that if in critical region, \underline{reject H\textsubscript{o}}.
\item Else \underline{fail to reject H\textsubscript{o}}

\end{itemize}

\begin{flushleft}
\textbf{t-Statistics and t-Tests}
\end{flushleft}
\begin{itemize}
\item <++>
\end{itemize}

\begin{flushleft}
\textbf{ANOVA}

\emph{insert table here}

\end{flushleft}
\end{flushright}
}

\end{tabular}
\end{scriptsize}
\begin{scriptsize}
\begin{tabular}{l | l}

\parbox{0.5\textwidth}{

\begin{normalsize}
\textbf{NPHT}
\end{normalsize}
\begin{itemize}

\item Parameter is $\mu$ [population mean]. $\mu\textsubscript{0} = \mu\textsubscript{1}$
\item $\bar{X}$ is sample mean. Under CLT, normal distribution at $\mu\textsubscript{0}$ for H\textsubscript{0} and $\mu\textsubscript{1}$ for H\textsubscript{1}.
\item We accept H\textsubscript{0} if \underline{\underline{not}} in CR.

\end{itemize}
\begin{normalsize}
\textbf{N-P Power Analysis}
\end{normalsize}
\begin{itemize}
\item Define parameter and its value under H\textsubscript{0} and H\textsubscript{1}
\item Define a test statistic and its sampling distribution under both hypotheses.
\item Use $\alpha$ to compute critical region
\item Compute power and compare to 80% threshold
\end{itemize}
\begin{normalsize}
\textbf{One-Sample T-Statistic [NP]}
\end{normalsize}
\begin{itemize}
\item If t\textsubscript{observed} in CR, then accept H\textsubscript{1}: $\mu = \mu\textsubscript{1}$. Else accept H\textsubscript{0}: $\mu = \mu\textsubscript{0}$
\end{itemize}
\begin{normalsize}
\textbf{Two-Tailed Test}
\end{normalsize}
\begin{itemize}
\item Take the upper and lower limit of the curve and the significance level ($\alpha$) is the cut off point of being \emph{statistically significant}. Treat as critical region. If in CR, then accept H\textsubscript{1}. Else accept H\textsubscript{0}.
\end{itemize}

\begin{flushleft}
\textbf{ANOVA}
\end{flushleft}
\begin{itemize}
\item Null Hypothesis: $H\textsubscript{0}: \mu\textsubscript{1} = \mu\textsubscript{2} = \mu\textsubscript{3} = \mu\textsubscript{i}$
\item If the variability BETWEEN the means ($\Delta x$) in the numerator is relatively large compared to the variance within the samples (internal spread) in the denominator, the ratio will be much larger than 1.
\item The samples then most likely do NOT come from a common population REJECT H\textsubscript{0}. (if at least one of the $\mu$ is not equal.)
\item ANOVA tests \underline{\textbf{CANNOT}} determine/make conclusions about all populations means ($\forall$), only at least one element in the set ($\mu \in \forall$)
\item \underline{Usage:} compare control group and observational studies of more than three populations.
\end{itemize}
}

&

\parbox{0.5\textwidth}{
\begin{normalsize}
\textbf{NHST}
\end{normalsize}
\begin{flushright}
\begin{itemize}

\item Define a parameter and it's value under H\textsubscript{0}.
\item Define an interval representing an inequality
\item Define a test statistic and its sampling distrubution under H\textsubscript{0}
\item Compute p-value. P-Value $\le$ sig level $\implies$ reject H\textsubscript{0} \& accept H\textsubscript{1}. P-Value > sig level $\implies$ fail to reject H\textsubscript{0}. Can only be >, < $\ne$.

\end{itemize}
\end{flushright}
\begin{normalsize}
\textbf{Two-Sided Test}
\end{normalsize}
\begin{flushright}
\begin{itemize}
\item \textbf{Neyman-Pearson}
\item Critical region is $\frac{1}{2}$ left tail and $\frac{1}{2}$ right tail of sampling distribution under H\textsubscript{0}. Power will $\downarrow$.
\item \textbf{NHST}
\item Find the "one-sided" p-value and double it.
\end{itemize}

\end{flushright}
\begin{normalsize}
\textbf{Matched Pairs t-Test}
\end{normalsize}
\begin{itemize}
\item Paired subjects receives their respective treatment or an individual gets two treatments. Also a subset of block design. 
\item H\textsubscript{0}: $\mu\textsubscript{d} = 0$ (no difference) and H\textsubscript{a}: $\mu\textsubscript{d} \ne 0$ (difference). 
\item If p-value $\le \alpha$, we reject H\textsubscript{0} \& accept H\textsubscript{a} conclude there is a difference. 
\item If p-value > significance level, we fail to reject H\textsubscript{0} cannot claim there is a difference. (We do not have any definitive truth to accept the null hypothesis)
\item \underline{Requirements:} large population, normal distribution, $\sigma$ is unknown. 
\end{itemize}
\begin{flushright}
\end{flushright}
}
\end{tabular}
\end{scriptsize}

\end{document}
