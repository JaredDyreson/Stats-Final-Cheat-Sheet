% author(s): Jared Dyreson
% school: California State University Fullerton, 2019
% first exam material condensed

\documentclass[6pt]{article}

\usepackage[letterpaper]{geometry}
\usepackage{mathtools}
\usepackage{amsmath}
\usepackage{amssymb}
\usepackage{lmodern}
\geometry{top=1.0cm, bottom=1.0cm, left=1.0cm, right=1.0cm}
\pagenumbering{gobble}

% horizontal line spanning the whole page

\newcommand{\HL}{\par\noindent\rule{\textwidth}{0.4pt}}

\begin{document}

\begin{footnotesize}

\begin{center}
\textbf{MATH-338 Final Cheat Sheet [Exam 1]}
\end{center}

\begin{flushleft}
\textbf{TERMS}
\end{flushleft}

\HL

\textbf{Probability Model:} a mathematical representation of a random event. 
\textbf{Sample Space:} a range of values of a random variable. 
\textbf{Probability:} a quantifiable amount for the likelihood of an event occurring. 
%\textbf{Axioms of Probability:} \textbf{1)} Probability of l
\textbf{Independent:} one event does not influence another. In conditional probability, this means $P(A \cap B) = P(A) \cdot P(B)$.
\textbf{Disjoint:} the probability of two specific events occurring is 0.
\textbf{Neither (example):} a person does not eat red meat \& is vegetarian.
\textbf{Both (example):} because you passed your driver's test, you have a driver's license.
\textbf{PMF:} $P(X = x) \rightarrow$ mapped value.
\textbf{Case:} elements in the collection and has one or more attributes.
\textbf{Collection (Population):} $\forall \in S$ ($S$ denotes set).
\textbf{Collection (Sample):} $\subset \in S$.
\textbf{Parameter:} any numerical quantity that characterizes a given population or some aspect of it (quantitative).
\textbf{Statistic:} a characteristic of a sample (qualitative).
\textbf{Distribution (Variable):} a description of the relative number of times each possible outcome will occur in a number of trails.
\textbf{Distribution (Sampling):} a description of the given sample observed.
\textbf{Bias:} the process to over-or-under estimate the value of a population.
\textbf{Biased Estimator:} difference between expected value and the true value of the parameter being estimated. (residual?)
\textbf{BINS:} \textbf{B}inary outcome, \textbf{I}ndependent trials, \textbf{N}umber of trials is fixed, \textbf{S}ame value of $p$ for all trials.
\textbf{Explanatory variable:} a type of independent variable (X value)
\textbf{Response variable:} dependent variable (y-value)
\textbf{Observational study:} not interacting with test subjects
\textbf{Experimental study:} manipulate environmental variables that affect test subjects.
\textbf{Levels of a Factor:} the number of variation of a given factor that were used in an experiment.
\textbf{Interacting variables:} two variables that interact with each other to produce an interaction effect (internal agent).
\textbf{Confounding variables:} a variable that influences both the dependent variable and independent variable, causing a spurious action (external agent).
\textbf{Placebo Effect:} a beneficial effect produced by a placebo treatment, which cannot be attributed to the properties of the placebo itself. This is generally used in the control group.
\textbf{Control:} a group that is placed in conditions that are seen as inert, not affecting the outcome of the study.
\textbf{Randomization:} does not lead to bias
\textbf{Replication:} the ability to do the study with other experimenters and get similar results
\textbf{Repetition:} do something over and over by the same experimenter getting similar results.
\textbf{Random Design:} anyone in the sample is likely to get a particular treatment and there are no predetermined groups.
\textbf{Block Design:} the experimenter divides the groups into based on a criteria then those groups are given specific treatments.
\textbf{Matched Pairs:} two subjects of similar characteristics are given two different treatments and the differences are compared.
\textbf{Single-Blind:} information is concealed from the patients in the study
\textbf{Double-Blind:} information is  concealed from both experimenters and the patients in the study.
\textbf{Conditional Probability:} an events outcome probability is dictated by previous events
\textbf{<++>:} <++>

\HL

\begin{flushleft}
\textbf{FORMULAS}
\end{flushleft}

\HL


\begin{tabular}{l | l}

\parbox{0.5\textwidth}{

\begin{itemize}

\item $\mu_{x} = \Sigma x \cdot p(x)$ [expected value]
\item $\sigma^{2}_{x} = \Sigma [x^{2} \cdot P(x)] - \mu^{2}_{x}$ [variance]
\item $\sigma_{x} = \sqrt{\Sigma [x^{2} \cdot P(x)] - \mu^{2}_{x}}$ [standard devitation]
\item $E(X \pm Y) = E(X) \pm E(Y)$ [expected value (transformation)]
\item $Var(X \pm Y) = Var(X) \pm Var(Y)$ [variance (transformation)]
\item $E(\hat{p}) = p$ [expected value (sample proportion)]
\item $\sigma_{\hat{p}}^2 = \frac{p(1-p)}{n}$ [variance (sample proportion)]
\item $\sigma_{\hat{p}} = \sqrt{\frac{p(1-p)}{n}}$ [standard deviation (sample proportion)]
\item $E(X) = \mu_{x} = nP$ [expected value (BRV)]
\item $Var(X) = n \cdot P \cdot (1-P)$ [variance (BRV)]
\end{itemize}

}

&

\parbox{0.5\textwidth}{

\begin{flushright}

\begin{flushleft}
\textbf{TWO WAY TABLE}
\end{flushleft}


\begin{flushleft}
 \begin{tabular}{||c | c | c||} 
 \hline
 Validity & Do not reject H\textsubscript{o} & Reject H\textsubscript{o} \\ [0.5ex] 
 \hline\hline
 H\textsubscript{o} is true & Correct decision & Type I Error $\alpha$ \\ 
 \hline
 H\textsubscript{o} is false & Type II Error $\beta$ & Correct decision \\ 
 \hline
\end{tabular}

\begin{itemize}
\item Type I Error: seeing a wolf when there is not a wolf
\item Type II Error: not seeing a wolf when there is a wolf
\end{itemize}


\begin{flushleft}
\textbf{DIAGNOSTIC TEST TABLE}
\end{flushleft}

 \begin{tabular}{||c | c | c | c||} 
 \hline
  & (+) & (-) & \\
 \hline
 + & TP & FN & Sensitivity \\ [0.5ex] 
 \hline
 - & FP & TN & Specificity \\ 
 \hline
 & PPV & NPV & \\
 \hline
\end{tabular}

\begin{itemize}
\item $PPV = \frac{TP}{TP+FP}$ \quad \quad $NPV = \frac{TN}{TN+FN}$
\item $SN = \frac{TP}{TP+FN}$ \quad \quad $SP = \frac{TN}{TN+FP}$
\end{itemize}

\end{flushleft}
\end{flushright}
}

\end{tabular}

\HL

\begin{flushleft}
\textbf{FRAMEWORKS}
\end{flushleft}

\HL


\begin{tabular}{l | l} 

\parbox{0.5\textwidth}{

\begin{flushleft}
\textbf{Neyman-Pearson Hypothesis Testing}
\end{flushleft}

\begin{itemize}
\item H\textsubscript{o}: $\mu = 0$
\item H\textsubscript{a}: $\mu = n$
\item Requires a rejection region, a small area where the null hypothesis should be rejected
\item If the observed value falls in the region, H\textsubscript{a} is true, reject H\textsubscript{o}, vice versa.
\end{itemize}

\begin{flushleft}
\textbf{Fisher's Significance Testing}
\end{flushleft}

\begin{itemize}
\item Select an appropriate test
\item Set up H\textsubscript{o}
\item Calculate the theoretical proabability of the results under H\textsubscript{o} (\emph{p})
\item If \emph{p} $= \alpha \therefore$ statistically significant
\item If \emph{p} $> \alpha \therefore$ statistically insignificant
\end{itemize}

}

&

\parbox{0.5\textwidth}{

\begin{flushright}

\begin{flushleft}
\textbf{Null Hypothesis Significance Testing}
\end{flushleft}

\begin{itemize}

\item H\textsubscript{o}: $\theta = \theta_{1}$ (if candy causes cancer, assume candy does not cause cancer and find counter arguments)
\item H\textsubscript{a}: $\theta [<,>,\ne] \theta_{1}$
\item Find its distribution under H\textsubscript{o}
\item Define a critical region such that if in critical region, \underline{reject H\textsubscript{o}}.
\item Else \underline{fail to reject H\textsubscript{o}}
\end{itemize}
\end{flushright}
}

\end{tabular}
\HL


\end{footnotesize}
\end{document}
